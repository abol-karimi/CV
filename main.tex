% LaTeX file for resume
% This file uses the resume document class (res.cls)

\documentclass[margin]{res}
\usepackage[top=17mm, bottom=21mm, left=20mm]{geometry}
% the margin option causes section titles to appear to the left of body text
\textwidth=5.5in % increase textwidth to get smaller right margin
%\resumewidth=5.2in
%\usepackage{helvetica} % uses helvetica postscript font (download helvetica.sty)
%\usepackage{newcent}   % uses new century schoolbook postscript font

\usepackage{multicol}
\setlength{\columnsep}{0mm}

\usepackage{fancyhdr}
\pagestyle{fancy}
\fancyhf{}
\renewcommand{\headrulewidth}{0pt}
\rfoot{\footnotesize{Page \thepage \hspace{1pt} of 3}}

\usepackage{enumitem}

\begin{document}
\vspace*{-10mm}
\name{Abolfazl (Abel) Karimi\\[12pt]} % the \\[12pt] adds a blank line after name

\begin{resume}

\section{Contact}
\begin{minipage}[t]{0.5\textwidth}
    Residence: Scarsdale, NY \\
    Personal Email: abel.karimi@proton.me \\
    https://linkedin.com/in/abolfazl-abel-karimi
\end{minipage}
\hspace{5mm}
\begin{minipage}[t]{0.5\textwidth}
    Mobile: +1 (309) 569-1670 \\
    Work Email: ak@cs.unc.edu \\
    https://github.com/abol-karimi
\end{minipage}


\section{Education}
PhD in Computer Science, 2019 - 2024 (expected), \\
University of North Carolina at Chapel Hill. \\
{\scriptsize Research: Simulation-based Testing of Autonomous Vehicles, \\
Advisor: Parasara Sridhar Duggirala}
 
PhD in Mathematics (left with an M.S. to move to UNC with my advisor), 2015 - 2018, \\
University of Connecticut, Storrs, Connecticut. \\
{\scriptsize Research: Mathematical Logic, Formal Verification of Cyber-Physical Systems, \\
Advisors: David Reed Solomon, Parasara Sridhar Duggirala.}

M.S. in Mathematics, 2013 - 2015, Western Illinois University, Macomb, Illinois. \\
{\scriptsize Reverse Mathematics (thesis), Mathematical Logic, Computability Theory.}

M.S. in Computer Science, 2008 - 2010, Yazd University, Iran. \\
{\scriptsize Algebraic Combinatorics (thesis), Scientific Computing. }

B.S. in Software Engineering, 2003 - 2008, University of Tehran, Iran.

Diploma in Mathematics and Physics, 1996 - 2003, Yazd, Iran. \\
{\scriptsize NODET (National Organization for Development of Exceptional Talents) Middle School and High School.}



%--- Work Experience ---
\section{Work Experience}
Systems Engineer Intern, May-November 2022; \\
Argo AI, LLC; Pittsburgh, Pennsylvania. \\
{\scriptsize Manager: Shawn Cook.}

Research Engineer Intern, Summer 2020; \\
Higharc, Inc; Durham, North Carolina. \\
{\scriptsize Supervisor: Peter Boyer, Director of Engineering.}

Teaching Assistant for ``Models of Languages and Computation", Spring 2023,\\
Department of Computer Science, University of North Carolina at Chapel Hill.

Teaching Assistant for ``Calculus I, II", ``Multivariable Calculus"; \\
Instructor of ``Elementary Discrete Mathematics"; 2015 - 2018,\\
Department of Mathematics, University of Connecticut.

Instructor of ``Intermediate Algebra" and ``Precalculus Algebra",\\
Teaching Support Assistant for Calculus and Algebra; 2013 - 2015,\\
Department of Mathematics, Western Illinois University.

Instructor of ``Logic", Fall 2012, \\
Faculty of Mathematics, Yazd University, Iran.



%------ Research interests -------
\section{Research \\ Interests}
\textbf{As an Experimentalist}:
Simulation-based testing of Autonomous Vehicles,\\
Machine Learning, Data Science, Reproducibility of scientific experiments.


\textbf{As a Logician}:
Formal Verification, Automated Reasoning, \\
Logical foundations of probability theory.



\newpage
%------ Publications -------
\section{Publications}
\underline{A. Karimi}, P. S. Duggirala, ``Diverse Traffic Scenario Generation for Autonomous Vehicles using Fuzzing," (in submission).

\underline{A. Karimi}, P. S. Duggirala, ``Automatic Generation of Test-cases of Increasing Complexity for Autonomous Vehicles at Intersections,"
International Conference on Cyber-Physical Systems, 2022.

\underline{A. Karimi}, P. S. Duggirala, ``Formalizing traffic rules for uncontrolled intersections,"
International Conference on Cyber-Physical Systems, 2020.

\underline{A. Karimi}, M. Goyal, P. S. Duggirala, ``Safety and progress proofs for a reactive planner and controller for autonomous driving," arXiv preprint arXiv:2107.05815, 2021.

M. R. Hooshmandasl, A. Shakiba, A. K. Goharshady, and \underline{A. Karimi}, ``S-Approximation: A New Approach to Algebraic Approximation," Journal of Discrete Mathematics, vol. 2014, Article ID 909684, 5 pages, 2014.

M. R. Hooshmandasl, \underline{A. Karimi}, M. Alambardar, B. Davvaz, ``Axiomatic systems for rough set-valued homomorphisms of associative rings", International Journal of Approximate Reasoning, Volume 54, Issue 2, February 2013, Pages 297-306, ISSN 0888-613X.


\section{Awards and Honors}
- Won the Generalized Racing Intelligence Competition (GRAIC) at CPS-IoT Week 2021.
- Our team ranked 1\textsuperscript{st} in ``F1/10 Autonomous Racing Competition", CPS-IoT Week 2019;\\
and ranked 2\textsuperscript{nd} in the competition at CPS Week 2018.\\
- Tuition Scholarship of Yazd University (2008-2010) and University of Tehran (2003-2008). \\
- Ranked 279\textsuperscript{th} among near 500,000 participants in nationwide university entrance exam (Konkoor-e Sarasari) for B.S. degree, 2003. \\
- Ranked 1\textsuperscript{st} in Ayandesazan National Olympiad of Iran among +110,000 participants, 1996.

\section{Committee Work}
Academic Integrity Committee member (Fall 2014),\\
Grade Appeals Committee member (Spring \& Fall 2014),\\
Department of Mathematics, Western Illinois University.


\section{Professional Membership}
Association for Computing Machinery (ACM), since 2017.\\
Institute of Transportation Engineers (ITE), since 2019.\\
Society of Automotive Engineers (SAE International), since 2020.

\section{Presentations}
Presented ``An Introduction to Modal Logic", S.I.G.M.A Seminar, Department of Mathematics, University of Connecticut, April 13, 2018.

Presented ``An Introduction to Reverse Mathematics", XV Graduate Student Conference in Logic, University of Wisconsin-Madison, April 26-27, 2014.

Presented ``Unavoidable Patterns on Infinite Words", XIV Graduate Student Conference in Logic, University of Illinois at Urbana-Champaign, April 20-21, 2013.

\section{Workshop}
Attended Logic Mentoring Workshop 2016, Thirty-First Annual ACM/IEEE Symposium on
Logic in Computer Science (LICS), Columbia University, July 9th, 2016.

\section{Summer School}
Attended the Ninth Summer School on Formal Techniques (May 18 - 25, 2019),
organized by SRI International, Menlo Park, CA.

Attended the Asian Initiative for Infinity (AII) Graduate Summer School in Singapore (July 15 - 26, 2013), organized by the Institute of Mathematical Sciences (IMS), National University of Singapore and funded by the John Templeton Foundation.


\section{Computer Skills}
\textbf{Programming}:
Python,
Bash,
C/C++,
JavaScript \& TypeScript,
Logic Programming,
SAT and SMT solving. \\
\textbf{Robotics}:
Unreal Engine,
CARLA,
SCENIC,
ROS (Robot Operating System).\\
\textbf{Mathematics Software}:
Matlab,
Mathematica,
Maple. \\
\textbf{Software Engineering}:
Git,
Docker,
Singularity,
SLURM.


\section{Languages}
English (Fluent); 
Farsi (Native); 
Arabic (Reading Knowledge).

% Test scores:
% \begin{itemize}
% \item
% TOEFL iBT, 29 November 2014 \\
% \hspace*{3mm} Total: 109 (Reading: 27, Listening: 29, Speaking: 26, Writing: 27)\\

% \vspace*{-3mm}
% \item
% GRE revised General, 19 November 2011\\
% \hspace*{3mm} Verbal: 152 (54\% below), Quantitative: 166 (92\% below), Writing: 3.0 (15\% below)
% \end{itemize}
% \vspace*{-3mm}
% Persian (Native); \\
% Reading Knowledge of Arabic.

%\newpage
%
%\section{Non-degree Education}
%{\bf Self-Studies} in Foundations and Philosophy of Mathematics and Logic (2011-2013) \\
%I spent this period on finding a discipline in which fundamental questions about the nature of this world are pursued.
%I did an extensive search about the research interests of mathematical logicians, philosophers and computer scientists around the world to obtain a richer perspective and choose my long-term research area confidently.
%
%Here is a selection of books and articles that I read (in this period) most or some parts of, and which helped me gain a clear idea of my mathematical pursuits:
%
%\begin{multicols}{2}
%\begin{itemize}
%\item ``Logic and Structure" \\ {\scriptsize by van Dalen,}
%\item ``Logic for Mathematicians" \\ {\scriptsize by Hamilton,}
%\item ``Logic and Computability" \\ {\scriptsize by Boolos, Burgess, and Jeffrey,}
%\item ``Elements of Set Theory" \\ {\scriptsize by Enderton,}
%\item ``The Joy of Sets" \\ {\scriptsize by Devlin,}
%\item ``Lambda-Calculus and Combinators" \\ {\scriptsize by Hindley and Seldin,}
%\item ``Computability and Unsolvability" \\ {\scriptsize by Davis,}
%\item ``Basic Proof Theory" \\ {\scriptsize by Troelstra and Schwichtenberg,}
%\item ``Modal Logic" \\ {\scriptsize by Blackburn, de Rijke and Venema,}
%\item ``Thinking about Mathematics" \\ {\scriptsize by Shapiro,}
%\item ``An Introduction to G\"{o}del's Theorems"  {\scriptsize by Smith,}
%\item ``The Myth of Hypercomputation" \\ {\scriptsize by Davis,}
%\item ``Toward objectivity in mathematics" \\ {\scriptsize by Simpson,}
%\item ``Partial Realizations of Hilbert's Program" {\scriptsize by Simpson,}
%\item ``Hilbert's Program Then and Now" \\ {\scriptsize by Zach,}
%\item ``Remarks on Finitism" \\ {\scriptsize by Tait.}
%\item ``On Computability" \\ {\scriptsize by Sieg,}
%\item ``Second Philosophy" \\ {\scriptsize by Maddy,}
%\item ``Meta Math!" \\ {\scriptsize by Chaitin,}
%\item ``Infinity" \\ {\scriptsize by Fletcher,}
%
%\end{itemize}
%\end{multicols}

%\section{References}
%Dr. Dinesh Ekanayake,\\
%Department of Mathematics/Graduate Advisor,\\
%Western Illinois University, Macomb IL.\\
%Email: DB-Ekanayake@wiu.edu \\
%Phone: 309-298-1239
%
%Dr. Clifton Ealy,\\
%Department of Mathematics,\\
%Western Illinois University, Macomb IL.\\
%Email: CF-Ealy@wiu.edu \\
%Phone: 309-298-1236
%
%Dr. Mohammadreza Hooshmandasl,\\
%Department of Computer Science, Faculty of Mathematics,\\
%Yazd University, Yazd, Iran.\\
%Email: hooshmandasl@yazd.ac.ir \\
%Phone: +98-353-8122690
\end{resume}
\end{document}
